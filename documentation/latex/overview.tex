%-------------------------------------------------------------------------------
% File: overview.tex
%       Epidemic Broadcast project documentation.
%
% Author: Marco Pinna, Rambod Rahmani, Yuri Mazzuoli
%         Created on 05/12/2020
%-------------------------------------------------------------------------------
\chapter{Overview}\label{overview}
To perform the analysis of the broadcast of a message that should reach as many
users as possible in a 2D floorplan, the following hypotheses were made:
\begin{itemize}
	\item the floorplan always has a rectangular shape and it is empty, (i.e.
	there are no obstacles such as walls or pillars in it);
	\item each user is considered point-like and does not move inside the
	floorplan;
	\item the transmission of a message is instantaneous and it happens at the
	beginning of every time slot; the whole apparatus can therefore be
	considered a \textit{Discrete Time System}.
\end{itemize}
Depending on the performance metrics to be analysed, different choices can be
made about the parameters to be adjusted. According to the specifications, the
main three metrics for this study are:
\begin{itemize}
	\item the broadcast time \colorbox{gray!30}{\large \texttt{T}} for a message
	to cover as many user as possible in the entire floorplan; the effective
	duration of the transmissions slots obviously has an effect
	on the total broadcast time, but it is only a scaling factor on the total
	number of slots; \texttt{T} can therefore be measured in terms of slots and
	converted to units of time accordingly.
	\item the percentage of covered users \colorbox{gray!30}{\large \texttt{U}};
	\item the number of collisions \colorbox{gray!30}{\large \texttt{C}}:
	collisions are detected by nodes, they happen when a node receives more than
	one message in the same time slot; in order to measure the number of
	collisions;
\end{itemize}
The following parameters have been identified: the transmission range of the
users, the \textit{per-slot} transmission probability, the floorplan size and
its shape, and the density of users in the floorplan per square metre.\\
\\
To be more detailed:
\begin{itemize}
	\item the radius of transmission \colorbox{gray!30}{\large \texttt{R}}: it
	represents the maximum distance between two users such that the message
	sent from one is detected by the other; it is the same for every user on
	the floorplan.\\
	$R$ clearly has a great impact on all
    the performance metrics: the greater this radius, the faster the message
    moves across the floorplan and the higher the number of users that can be
    reached; on the other hand, a greater radius is likely to cause more
    collisions than a smaller one. \\
	 Realistic values for $R$ have been taken from Bluetooth Low
	Energy standard and range from a minimum of $5$ metres to a maximum of $20$
	metres.
	\item the \textit{per-slot} transmission probability
	\colorbox{gray!30}{\large \texttt{p}}: it is the success probability for the
	Bernoullian random variable associated to the transmission. As a
	probability, it can assume values between 0 and 1.\\
	The higher the transmission probability $p$, the faster a message ”moves away” from
a user; at the same time, a high transmission probability implies a high collision
probability in a local area where two or more nodes are transmitting;
    \item the length of the side of the floorplan rectangle
    \colorbox{gray!30}{\large \texttt{L}}; \\
    A bigger floorplan area, all else being equal, will require a longer
    time to be covered entirely by the broadcast;
	\item the \textit{aspect ratio} of the floorplan rectangle
	\colorbox{gray!30}{\large \texttt{a}}, defined as the ratio between the
	longer side $L$ and the shorter side $W$; because of the radial symmetry of
	the transmission phenomenon, there is no actual need to consider values for
	\texttt{a} lower than 1.\\
	A very long and very narrow floorplan will probably cause less collisions than a
square one with the same area, as the average number of users in the collision
range of a random user decreases; on the other hand, the performance of this type of
scenario will be highly influenced by the position of the starting node;

%TODO remove? Reprhase? We are already specifying that we only tweaked p and R
%	this parameter is fixed at 1 in our simulation model, 
%	so we are considering only square floorplans; this will allow the analysis
%	to be focused on more important parameters, but still in a quite common
%	scenario;
%TODO number of users is defined as number of users? Redundant? Remove definition? Rephrase?
	\item number of users \colorbox{gray!30}{\large \texttt{N}}, defined
	as the number of users dropped on the floorplan;
	\item users population density \colorbox{gray!30}{\large \texttt{d}},
	defined as the number of users per square meter; for this parameters, we
	decided to study two opposite density scenarios
	(high: 1/$m^2$ and low 0.1/$m^2$).
\end{itemize}

For this study, the two main parameters were \colorbox{gray!30}{\large \texttt{R}} and \colorbox{gray!30}{\large \texttt{p}}. Full sweeps for different values of both were made. \\
\colorbox{gray!30}{\large \texttt{a}} was kept constant and equal to 1 (i.e. the floorplan is always a square) and \colorbox{gray!30}{\large \texttt{L}} was either $10 m$ (``small" configuration) or $100 m$ (``big" configuration). \colorbox{gray!30}{\large \texttt{N}} was also kept constant and equal to 100.
