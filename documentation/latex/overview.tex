%-------------------------------------------------------------------------------
% File: overview.tex
%       Epidemic Broadcast project documentation.
%
% Author: Marco Pinna, Rambod Rahmani, Yuri Mazzuoli
%         Created on 05/12/2020
%-------------------------------------------------------------------------------
\chapter{Overview}\label{overview}
To perform the analysis of the broadcast of a message that should reach as many
users as possible in a 2D floorplan, the following hypotheses were made:
\begin{itemize}
	\item the floorplan always has a rectangular shape and it is empty, (i.e.
	there are no obstacles such as walls or pillars in it);
	\item each user is thought as a point mass and does not move inside the
	floorplan;
	\item the transmission of a message is instantaneous and it happens at the
	beginning of every time slot; the whole apparatus can therefore be
	considered a \textit{Discrete Time System}.
\end{itemize}
Depending on the performance metrics to be analysed, different choices can be
made about the parameters to be adjusted. According to the specifications, the
main three metrics for this study are:
\begin{itemize}
	\item the broadcast time \colorbox{gray!30}{\large \texttt{T}} for a message
	to cover as many user as possible in the entire floorplan; the effective duration 
	of the transmissions slots obviously has an effect
	on the total broadcast time\footnote{The total amount of time required for a
	broadcast message to cover the entire floorplan.}, but it is only a scaling
	factor on the total number of slots. \texttt{T} can therefore be measured in
	terms of slots and converted to units of time accordingly.
	\item the percentage of covered users \colorbox{gray!30}{\large \texttt{U}};
	\item the number of collisions \colorbox{gray!30}{\large \texttt{C}}; collisions
	are detected by nodes, when they try to receive more than one message in the same slot;
	in order to mesure the number of collisions, we consider a single event of collision
	every time a node detect one of them; every statistical analysis is done taking this
	assumption.
\end{itemize}

The following parameters have been identified: the transmission range of the
users, the \textit{per-slot} transmission probability, the floorplan size and
its shape, and the density of users in the floorplan per square metre.\\
\\
To be more in detailed:
\begin{itemize}
	\item the radius of transmission \colorbox{gray!30}{\large \texttt{R}}: it
	represents the maximum distance between two users such that the message
	sent from one is detected by the other; it is the same for every user on
	the floorplan; realistic values for $R$ have been taken from Bluetooth Low
	Energy standard and range from a minimum of $5$ metres to a maximum of $50$
	metres.
	\item the \textit{per-slot} transmission probability
	\colorbox{gray!30}{\large \texttt{p}}: it is the success probability for the
	Bernoulli random variable associated to the transmission; as a probability, it 
	can assume values between 0 and 1;
    \item the width of the side of the floorplan rectangle
    \colorbox{gray!30}{\large \texttt{L}};
	\item the \textit{aspect ratio} of the floorplan rectangle
	\colorbox{gray!30}{\large \texttt{a}}, defined as the ratio between the
	longer side $L$ and the shorter side $W$; because of the radial symmetry of the
	transmission phenomenon, there is no actual need to consider values for
	\texttt{a} lower than 1; this parameter is fixed at 1 in our simulation model, 
	so we are considering only sqare floorplans; this will allow us to focus the analysis
	on other parameters, but still considering a quite common scenario;
	\item number of users \colorbox{gray!30}{\large \texttt{N}}, defined
	as the number of users in the floorplan;
	\item users population density \colorbox{gray!30}{\large \texttt{d}}, defined
	as the number of users per square meter. We choose to study two opposite 
	density scenarios (high: 1/$m^2$, low 0.1/$m^2$) 
\end{itemize}
The rationales behind the choices of the parameters were the following:
\begin{itemize}
    \item the transmission radius clearly has a great impact on all the
    performance metrics; the greater the radius, the faster the message moves
    across the floorplan and the higher the number of users that can be reached;
    on the other hand, a greater radius is likely to cause more collisions than
    a smaller one;
    \item the higher the transmission probability, the faster a message "moves
    away" from a user; at the same time, a high transmission probability implies
    a high collision probability in a local area where two or more nodes are
    transmitting;
    \item a bigger floorplan area, all else being equal, will require a longer
    time to be covered entirely by the broadcast message;
    \item a very long and very narrow floorplan will probably cause less
    collisions than a square one with the same area, as the average number of
    users in the collision range of other users decreases; on the other hand, the 
	performance of this type of scenario will be higly influenced by the position of the 
	starting node;
    \item in order to exploit various values of  population density \texttt{d}
	we choose to fix the total users \texttt{N}=100, then we choose the 2 values for $L$: 
	$10$ metres and $100$ metres.
\end{itemize}


