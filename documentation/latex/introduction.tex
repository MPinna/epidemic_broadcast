%-------------------------------------------------------------------------------
% File: introduction.tex
%       Epidemic Broadcast project documentation.
%
% Author: Marco Pinna, Rambod Rahmani, Yuri Mazzuoli
%         Created on 05/12/2020
%-------------------------------------------------------------------------------
\chapter{Introduction}
In what follows the study on the broadcast of an epidemic message is carried
out.\\
The specifications are detailed in the following:
\begin{displayquote}
    \begin{specifications}
    {\Large Epidemic broadcast}\\
    Consider a 2D floorplan with \textit{N} users randomly dropped in it. A
    random user within the floorplan produces a \textit{message}, which should
    ideally reach all the users as soon as possible. Communications are
    \textit{slotted}, meaning that on each slot a user may or may not relay the
    message, and a message occupies an entire slot. A \textit{broadcast radius
    R} is defined, so that every receiver who is within a radius \textit{R} from
    the transmitter will receive the message, and no other user will hear it. A
    user that receives more than one message in the same slot will not be able
    to decode any of them (\textit{collision}). Users relay the message they
    receive \textit{once}, according to the following policy
    (\textit{p-persistent relaying}): after the user successfully receives a
    message, it keeps extracting a value from a Bernoullian RV with success
    probability \textit{p} on every slot, until it achieves success. Then it
    relays the message and stops. A sender does not know (or cares about)
    whether or not its message has been received by its neighbors.\\
    \\
    Measure at least the broadcast time for a message in the entire floorplan,
    the percentage of covered users, the number of collisions.\\
    \\
    In all cases, it is up to the team to calibrate the scenarios so that
    meaningful results are obtained.
    \end{specifications}
\end{displayquote}
The work is organized as follows:
\begin{itemize}
    \item Firstly, in chapter \ref{ch:overview}, an initial overview and a presentation of the problem are
    given. Here, meaningful parameters to be tweaked and useful scenarios are
    identified and some considerations about them are made.
    \item Secondly, in chapter \ref{ch:modelling}, a graph-based modelling technique, commonly used in
    literature, is proposed and some simplified scenarios are analysed.
    \item In chapter \ref{ch:simulator} the development of the simulator is
    described and the results of its validation are presented.
    \item Chapter \ref{ch:simulation} concerns the full simulation and performance
    evaluation of the system.
	\item Lastly, in chapter \ref{ch:conclusions} conclusions are drawn about the results of the study and some suggestion are given regarding further research that could possibly be done on the subject.
\end{itemize}

\noindent
The entire codebase (both the  Omnet++ files and the scripts used for data analysis) is available at \url{https://github.com/MPinna/epidemic_broadcast} .