%-------------------------------------------------------------------------------
% File: simulator.tex
%       Epidemic Broadcast project documentation.
%
% Author: Marco Pinna, Rambod Rahmani, Yuri Mazzuoli
%         Created on 05/12/2020
%-------------------------------------------------------------------------------
\chapter{Simulator}\label{simulator}
In order to obtain experimental results for the presented scenarios, a simulator
was built using OMNeT++. This will allow us to reproduce the different scenarios
with different values for the identified parameters.\\
\section{Omnet++ and INET framework}
OMNeT++ is an extensible, modular, component-based C++ simulation library and
framework, primarily for building network simulators\footnote{https://omnetpp.org/}.\\
The INET Framework is an open-source model library for the OMNeT++ simulation
environment. It provides protocols, agents and other predefined models for
researchers and students working with communication networks. INET is especially
useful when designing and validating new protocols, or exploring new or exotic
scenarios\footnote{https://omnetpp.org/download-items/INET.html}.\\
OMNeT++ is a library and a framework, and can be used with the dedicated IDE.
Not only it allows for development of the simulator itself, but also to export
simulation results and to inspect simulation behaviour with a graphical user
interface. Exploiting C++ compiler optimizations, it can achieve low simulation
duration. Netowks are composed by modules; there are two types for a module:
simple module and compound module (the last one can contain other modules).
INET is an extension of OMNET++, oriented to recreating a network simulation
environment, with the capability of reproducing the activity of a wireless
communication system across multiple nodes. It contains ready to use definitions
and implementations of network related modules.
\section{Network architecture}
The network based architecture is composed by an Array of Host modules, a
visualizer (the Integrated visualizer) and a radioMedium (UnitDiskRadioMedium);
% TODO: immagine rete
\begin{itemize}
    \item \textbf{UnitDiskRadioMedium} is a compound module provided by INET. It can reproduce a the behaviour of the wireless communication 
    channel with various levels of abstraction
    \item \textbf{Integrated visualizer} is a compound module provided by INET. It's resposible for the visual representation of modules properties 
    and events in the graphic user interface.
    \item \textbf{Host} is a compound module representing a node in our network environment. 
\end{itemize}
Every Host contains 4 submodules:
% TODO: immagine interno modulo host
\begin{itemize}
    \item \textbf{Mobility} module privided by INET manage the position of the parent module (Host); it allow various types of movements,
        but we are going to use it only for the initial random placement of the nodes, then they will remain static.
    \item \textbf{interface Table} module is provided by INET and is required for correct operation of radioMedium module.
    \item \textbf{wlan} module is the wireless interface that allow nodes to communicate with each others. It's a compund module of type 
        \textbf{AckingWirelessInterface}, which is the simplest wireless interface provided by INET.
    \item \textbf{ProcUnit} is the processing unit, that implement the node behaviour when a message arrive. It's connected to
    the wlan module in order to receive and send messages through that.
\end{itemize}

The processing unit realize behaviours of nodes: LISTENING, TRANSMITTING, SLEEPING; it also collect statistics, emitting signals when 
a message is correctly received and when it's sent out; the wlan module emit a signal every time a collision is detected, than the broken
message is dropped instead of forwarded to the processing unit. 
