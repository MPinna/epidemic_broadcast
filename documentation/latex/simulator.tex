%-------------------------------------------------------------------------------
% File: simulator.tex
%       Epidemic Broadcast project documentation.
%
% Author: Marco Pinna, Rambod Rahmani, Yuri Mazzuoli
%         Created on 05/12/2020
%-------------------------------------------------------------------------------
\chapter{Simulator}\label{simulator}
In order to obtain experimental results for the presented scenarios, a simulator
was built using OMNeT++ with the support of the INET framework. This allowed us
to reproduce the different scenarios, presented in the previous chapters, with
different values for the identified parameters.\\
\section{Omnet++ and INET framework}
OMNeT++ is an extensible, modular, component-based C++ simulation library and
framework, primarily for building network simulators\footnote{https://omnetpp.org/}.\\
The INET Framework is an open-source model library for the OMNeT++ simulation
environment. It provides protocols, agents and other predefined models for
researchers and students working with communication networks. INET is especially
useful when designing and validating new protocols, or exploring new or exotic
scenarios\footnote{https://omnetpp.org/download-items/INET.html}.\\
\\
OMNeT++ is a library and a framework, and can be used with the dedicated IDE.
Not only it allows for development of the simulator itself, but also to export
simulation results and to inspect simulation behaviour with a graphical user
interface. Exploiting C++ compiler optimizations, it can achieve the lowest
simulation duration possible. Networks are composed by modules; there are two
types of modules: simple module and compound module (which can contain other
modules itself).
INET, on the other hand, is an extension of OMNeT++, dedicated to recreating 
network simulation environments, with the capability of reproducing the activity
of a wireless communication system across multiple nodes. It contains ready to
use definitions and implementations of network related modules.\\
The choice was made to use the INET framework in order to avoid spending too
much time on the coding side and therefore be able to focus more on other
aspects such as the problem modeling and analysis.
\section{Network architecture}
The network based architecture is composed by an Array of Host modules, an
Integrated visualizer (visualizer) and a UnitDiskRadioMedium (radioMedium);
\begin{figure}[H]
    \begin{center}
        \includegraphics[scale=0.35]{img/floorplan.png}
        \caption{Floorplan.ned}
        \label{fig:single_queue}
    \end{center}
    \vspace*{-0.8cm}
\end{figure}
\begin{itemize}
    \item \textbf{UnitDiskRadioMedium} is a compound module provided by INET.
    This radio medium model provides a very simple but fast and predictable
    physical layer behavior. It must be used in conjunction with the
    \texttt{UnitDiskRadio} model. It can simulate the behaviour of the wireless
    communication channel with various levels of abstraction.
    \item \textbf{Integrated visualizer} is a compound module provided by INET.
    It's resposible for the visual representation of modules properties and
    events in the graphic user interface.
    \item \textbf{Host} is the compound module developer for representing a node
    in our network environment.
\end{itemize}
The \texttt{Host} module extends the NodeBase module defined by INET. This
module contains the most basic infrastructure for network nodes that is not
strictly communication protocol related. The following diagram shows usage
relationships between types:
\begin{figure}[H]
    \begin{center}
        \includegraphics[scale=0.26]{img/nodebase.png}
        \caption{NodeBase Diagram.}
        \label{fig:single_queue}
    \end{center}
    \vspace*{-0.8cm}
\end{figure}
\begin{figure}[H]
    \begin{center}
        \includegraphics[scale=0.35]{img/host.png}
        \caption{host.ned}
        \label{fig:single_queue}
    \end{center}
    \vspace*{-0.8cm}
\end{figure}
\begin{itemize}
    \item the \texttt{mobility} module privided by INET manages the position of
    the parent module \texttt{Host}; it allows various types of movements, but
    we are going to be using it only for the initial random placement of the
    nodes, then all nodes will be stationary.
    \item the \texttt{interfaceTable} module is provided by INET and is required
    for correct operation of the radioMedium module.
    \item the \texttt{wlan} module is the wireless interface that allows nodes
    to communicate with each others. It's an \texttt{AckingWirelessInterface}
    compound module, which is the simplest wireless interface provided by INET.
    \item the \texttt{status} module is provided by INET as well and is required
    to shutdown and restart network interfaces.
    \item the \texttt{procUnit} module is the custom made processing unit, that
    implements the node behaviour when a message arrives. It's connected to the
    \texttt{wlan} module in order to be able to receive and send messages.
    \item \texttt{energyStorage}, \texttt{energyManagement} and
    \texttt{energyGenerator} are modules inherited from \texttt{NodeBase} but
    they are not instantiated as we don't need to model energy related
    behaviours.
\end{itemize}
The \texttt{wlan} module is in charge of checking each and every message for
collisions, and drop broken packets instead of forwarding them to the processing
unit. The processing unit \texttt{ProcUnit} implements the behaviours of the
nodes; it handles the broadcast message when received, and then its
retransmission when the random variable extraction results in a success. Finally
it shuts down the network interface, preventing it from receiving any messages
or provoking collisions. The network interface is turned off also for the entire
duration of the RV extractions.
\section{Parameters and Statistics}
During the simulation, signals are used to collect the statistics. They are all
collected by the \texttt{Flooplan} module:
\begin{itemize}
    \item The \texttt{wlan} module emits a signal every time a collision is
    detected; this signal is collected by the
    \texttt{packetDropIncorrectlyReceived} statistic of the same module; we are
    interested in the total number of collisions detected by each node.
    \item The \texttt{ProcUnit} module emits $2$ signals when initialized,
    \texttt{hostX} and \texttt{hostY}, collected, respectively, by the
    \texttt{hostXstat} and \texttt{hostYstat} statistics; those are the
    coordinates of the parent node in the floorplan.
    \item The \texttt{ProcUnit} module emits the \texttt{timeCoverage} singal as
    well, collected in the \texttt{timeCoverageStat} statistic; this is a
    vector containing, for each node that received the broadcast message, the
    number of the time slot when the broadcast message was actually received; at
    the end of the simulation, it's size represents the number of covered nodes.
\end{itemize}
Most significant parameters set up in the initialization file
(\texttt{floorplan.ini}) are reported below:
\begin{itemize}
    \item \texttt{Floorplan.host[*].procUnit.slotLength = 1}
    \item \texttt{Floorplan.host[*].procUnit.p = 1}
\end{itemize}
%snapshot del file ini con gli sweep per p ed r, e altri parametri più significativi 
\section{Design Choices and Optimizations}
Using the INET framework for the development of the simulator allowed us to
make use of pre-built modules for modeling wireless communications; for example,
collision detection and statistics collection is already implemented by INET
modules. During the development we choose for every aspect the optimal level of
abstraction for our purposes, but it's possible to model other aspects just 
changing the types of INET modules used, or by adding new ones. We voluntarily
avoided taking into account phenomena like pathloss and node movement, and we 
restricted our considerations to a discrete time scenario. However, modeling
continuous time scenarios can be done easily by changing few INET modules types
and attributes.\\
INET modules have also pre-built optimization structures, that become
indispensable when the number of hosts become larger; in order to make the
simulator ready for high complex scenarios we used an the \texttt{neighborCache}
structure offered by the \texttt{radioMedium} module. This module is in charge
of storing proximity information of each and every node, in order to speed up
message delivery. Setting the type of this module to \texttt{GridNeighborCache}
it's possible to reduce the time needed for a simulaton with more than
$2000$ devices dropped on the floorplan, by a factor of $10$; we also found that
this type of cache (with the right value for the \texttt{cellSize} parameter) is
the best trade-off between speed and memory occupancy, for this type of
workload\footnote{https://doc.omnetpp.org/inet/api-current/neddoc/inet.physicallayer.contract.packetlevel.INeighborCache.html}.  

\section{Validation}
To ensure the correctness of the simulator and the meaningfulness of the results, the simulator has been validated by means of two simplified scenarios, namely the single queue configuration and the star configuration, which are discussed in \ref{ssec:singlequeue} and \ref{ssec:star2} respectively.

\subsection{Single queue validation}

\begin{figure}[H]
    \begin{center}
        \includegraphics[scale=0.75]{img/singleQueueGUI.png}
        \caption{Validation configuration with 12 hosts placed on a line}
        \label{fig:single_queueGUI}
    \end{center}
    \vspace*{-0.8cm}
\end{figure}

In this configuration, host[0] always broadcasts during the first slot and then the message travels along the queue, with a total of 10 hops. On every hop, the per-slot probability of successful transmission is $p$, which implies an average number of attempts equal to $\frac{1}{p}$. Therefore, the expected coverage time is

\begin{equation}
    E[T] = 1 + 10 \cdot \frac{1}{p}
    \label{eq:singleQueueValidationAvgT}
\end{equation}

Validation was performed with 9 different configuration, one for  each value of $p$ ranging from $0.1$ to $0.9$, with 200 repetitions each.
The following results were obtained:

\begin{center}
\begin{tabular}{ | m{1cm} | m{5cm}| m{5cm} | }
\hline
$p$& Expected value & Experimental value (mean, interval for 95\% confidence)\\
\hline
$0.1$&$101.0$&$\textbf{99.68}, 95.55, 103.81$\\
\hline
$0.2$&$51.0$&$\textbf{49.9}, 47.93, 51.86$\\
\hline
$0.3$&$34.33$&$\textbf{34.17}, 32.88, 35.46$\\
\hline
$0.4$&$26.0$&$\textbf{25.79}, 24.93, 26.65$\\
\hline
$0.5$&$21.0$&$\textbf{20.84}, 20.24, 21.44$\\
\hline
$0.6$&$17.67$&$\textbf{17.42}, 17.0, 17.84$\\
\hline
$0.7$&$15.29$&$\textbf{15.2}, 14.87, 15.52$\\
\hline
$0.8$&$13.5$&$\textbf{13.48}, 13.25, 13.72$\\
\hline
$0.9$&$12.11$&$\textbf{12.14}, 11.97, 12.3$\\
\hline
\end{tabular}
\end{center}

As can be seen in the table above %add label to table and reference it?
the experimental results are consistent with the theoretical expected value, with a 95\% confidence level. %rephrase?
\subsection{Star 5-to-1}