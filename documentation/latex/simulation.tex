%-------------------------------------------------------------------------------
% File: simulation.tex
%       Epidemic Broadcast project documentation.
%
% Author: Marco Pinna, Rambod Rahmani, Yuri Mazzuoli
%         Created on 05/12/2020
%-------------------------------------------------------------------------------
\chapter{Simulation}\label{simulation}
As we previously described, we decided to consider a population of $100$ users
in two different scenarios:
\begin{itemize}
    \item \textbf{Small}: a $10$m x $10$m floorplan, with the transmission
    radius ranging from $1$m to $4.5$m, with a step of $0.5$m, and a Bernullian
    RV with success probability $p$ ranging from $0.05$ to $0.95$ with steps of
    $0.05$;
    \item \textbf{Big}: a $100$m x $100$m floorplan, with the transmission
    radius ranging from $1$m to $19$m, with a step of $1$m, and a Bernoullian RV
    with success probability $p$ ranging from $0.1$ to $0.9$ with steps of $0.1$;
\end{itemize}
In what follows, for each of the scenarios, the floorplan coverage, the
broadcast duration and the number of collisions.\\
In order to achieve statistically significant results with a minimum accuracy of
$90$\%, we decided to repeat the same scenario with the same parameters for
$200$ times, and then computed mean and median values of the performance
indexes, along with their confidence intervals.
\section{Big}\label{big}
\subsection{Coverage}
Figure \ref{fig:floorplancoverage1} shows the floorplan coverage achieved as a
function of the Bernullian RV success probability $p$, for different values of
the transmission radius $R$. As we can observe, the coverage reached is maximum
for lower values of $p$; this is due to the fact that a low transmission
probability minimizes the number of collisions, increasing the number of correct
transmissions. For values of $R < 8$ we observe an almost linear behaviour:
collisions are almost absents with such a small transmission radius, therefore,
variations of the value of $p$ do not play a fundamental role as far as it
concerns the final reached coverage. For $R \geq 8$ the achievable coverage
decreases with the increase of $p$, but quite slowly and also it does not
degenerate to $0$; instead it reaches a value between $10$\% and $40$\% of the
maximum.
\begin{figure}[H]
    \begin{center}
        \includegraphics[scale=.42]{img/Big_CovP_median.pdf}
    \end{center}
    \vspace*{-0.5cm}
    \caption{Floorplan coverage as a function of $p$ for different values of $R$}
    \label{fig:floorplancoverage1}
\end{figure}
\noindent
For values of $R$ between $11$ and $15$, the number of collisions has a strong
effect on the coverage percentage; this effect tends to decrease for larger
values of $R$; we know as a matter of fact that for $R \to L\sqrt{2}$ the
coverage tends to $100$\%, independently of any other parameter.\\
\\
Figure \ref{fig:floorplancoverage2} shows the floorplan coverage achieved as a
function of the transmission radius $R$, for different values of the Bernullian
RV success probability $p$. As expected, accordingly with the previous plot, the
final coverage of the flooplan is deeply influenced by the transmission range $R$.
\begin{figure}[H]
    \begin{center}
        \includegraphics[scale=.75]{img/Big_CovRange_mean.pdf}
    \end{center}
    \vspace*{-0.5cm}
    \caption{Floorplan coverage as a function of $R$ for different values of $p$}
    \label{fig:floorplancoverage2}
\end{figure}
\noindent
In this second plot we can observe that the coverage increases exponentially
with the transmission range. When the transmission range becomes large
($> 16$m), the coverage tends to $100$\%. We recognized the shape of the
sigmoid\footnote{https://mathworld.wolfram.com/SigmoidFunction.html} function,
which can be express as an hyperbolic tangent ($\tanh$); this function is always
increasing and can be manipulated to remain between $0$ and $1$ (like the
parameter we want to fit). A good fit for this curve is given by:
\begin{equation*}
    C = \frac{1+\tanh(aR+b)}{2}
\end{equation*}
where $C$ is the floorplan coverage, $R$ the transmission radius, and $a$ and
$b$ depend on $p$ as shown in the following table:
\begin{center}
\begin{tabular}{ | m{1cm} | m{4cm}| m{4cm} | }
\hline
$p$&$a$&$b$\\
\hline
$0.1$&$0.3628314535334378$&$4.356732935967713$\\
\hline
$0.2$&$0.3572920723369711$&$4.3206795324164835$\\
\hline
$0.3$&$0.34371134773480694$&$4.193385575628803$\\
\hline
$0.4$&$0.3217942874156316$&$3.9814759531755515$\\
\hline
$0.5$&$0.3091448275788839$&$3.88645016495332$\\
\hline
$0.6$&$0.2809863550485405$&$3.6000814495973996$\\
\hline
$0.7$&$0.25932462296148107$&$3.3996924196436975$\\
\hline
$0.8$&$0.23603288616377227$&$3.1648817545125527$\\
\hline
$0.9$&$0.20917405109220505$&$2.929033169711656$\\
\hline
\end{tabular}
\end{center}
\subsection{Duration}
This following figure shows the plot of the duration of the simulation (in
seconds) as a function of $p$, for different values of $R$.
\begin{figure}[H]
    \begin{center}
        \includegraphics[scale=.51]{img/Big_DurP_median.pdf}
    \end{center}
    \vspace*{-0.5cm}
    \caption{Simulation duration as a function of $p$ for different values of $R$}
    \label{fig:floorplancoverage3}
\end{figure}
\noindent
The simulation duration increases for lower values of $p$; this behaviour can be
explained as a consequence of two main factors:
\begin{itemize}
    \item the probability of retransmission is low, so nodes will spend most
    time slots waiting, without actually transmitting;
    \item with a smaller number of collisions, a higher number of nodes is
    reached; reaching a higher number of nodes requires a higher number of
    retransmissions.
\end{itemize}
We tried to interpolate the curves shown in Figure \ref{fig:floorplancoverage3}
with an hyperbolic function, because it fits well the parameters we want to
model; first of all, the hyperbole tends to infinity as $p$ tends to $0$, and
this is coherent with the reality, because if the probability of retransmission
becomes low, then the duration keeps increasing. If instead $p$ gets close to
$1$, the duration time tends to its minimum. We fit those shapes using:
\begin{equation*}  
    D = \frac{a}{P}+b 
\end{equation*}
where $D$ is the duration of the simulation, $p$ the probability of
retransmission, and $a$ and $b$ depend on $R$ as specified in the following
table:
\begin{center}
\begin{tabular}{ | m{1cm} | m{5cm}| m{5cm} | }
\hline
$R$&$a$&$b$\\
\hline
$1$&$2.4046845625846913$ e-09&$0.999999999244136$\\
\hline
$2$&$2.4046845625846913$ e-09&$0.999999999244136$\\
\hline
$3$&$0.7313966348173997$&$0.9495446821974095$\\
\hline
$4$&$1.3471717522270503$&$0.9454326532617134$\\
\hline
$5$&$1.7744177255115228$&$1.08724762049118$\\
\hline
$6$&$4.652279267805434$&$1.0487610714204172$\\
\hline
$7$&$9.586489452244717$&$1.088347295851843$\\
\hline
$8$&$15.841670397411658$&$1.0643797063994824$\\
\hline
$9$&$28.264640360031194$&$0.49558107430696624$\\
\hline
$10$&$43.141734847778814$&$0.22815570364285975$\\
\hline
$11$&$64.55944043148898$&$-0.07517936077209139$\\
\hline
$12$&$86.56257725232254$&$-0.09919809415424388$\\
\hline
$13$&$89.72686483087844$&$1.2578386401845827$\\
\hline
$14$&$82.15919574473708$&$3.0965825999956778$\\
\hline
$15$&$67.95268283187643$&$5.415446394742158$\\
\hline
$16$&$56.442159145137495$&$6.991880399220213$\\
\hline
$17$&$45.71412729773696$&$7.548465010848107$\\
\hline
$18$&$39.164500694866284$&$7.991652317096266$\\
\hline
$19$&$29.71836400235831$&$9.075854630828385$\\
\hline
\end{tabular}
\end{center}
To be able to compute the correct values the first two entries for $a$ and $b$,
a larger dataset is needed; as a matter of fact, if the transmission range is
too short nodes can not communicate: they are not in reach by each others.\\
\\
The following plot shows the duration of the simulation (in seconds) as a
function of $R$, for different values of $p$.
\begin{figure}[H]
    \begin{center}
        \includegraphics[scale=.7]{img/Big_DurRange_mean.pdf}
    \end{center}
    \vspace*{-0.5cm}
    \caption{Simulation duration as a function of $R$ for different values of $p$}
    \label{fig:floorplancoverage4}
\end{figure}
\noindent
The duration of the simulation tends to increase with the transmission range
reaching a peak around $13$m; the maximum value depends on the probability of
retransmission ($p$);
This is an interesting result, and can be explained by:?%todo
\subsection{Collisions}
This plot show the Number of collisions in function of P, for different values of R.
\begin{figure}[H]
    \begin{center}
        \includegraphics[scale=.4]{img/Big_CollP_median.pdf}
    \end{center}
    \vspace*{-0.5cm}
    \caption{Caption this}
    \label{fig:floorplancoverage5}
\end{figure}
\begin{figure}[H]
    \begin{center}
        \includegraphics[scale=.7]{img/Big_CollRange_median.pdf}
    \end{center}
    \vspace*{-0.5cm}
    \caption{Caption this}
    \label{fig:floorplancoverage6}
\end{figure}
