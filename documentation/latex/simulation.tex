%-------------------------------------------------------------------------------
% File: simulation.tex
%       Epidemic Broadcast project documentation.
%
% Author: Marco Pinna, Rambod Rahmani, Yuri Mazzuoli
%         Created on 05/12/2020
%-------------------------------------------------------------------------------

\chapter{Simulation}\label{simulation}
As we previusly described, we decided to consider a population of 100 users in two different scenarios:
\begin{itemize}
    \item \textbf{Small}: 10mx10m floorplan, with the transmission range going from 1m to 4.5m, with 0.5m steps, and
        the Bernullian base p going from 0.05 to 0.95 with 0.05 steps; 
    \item \textbf{Big}: 100mx100m floorplan, with the transmission range going from 1m to 19m, with 1m steps, and
        the Bernullian base p going from 0.1 to 0.9 with 0.1 steps; 
\end{itemize}
In order to acheive meaningfull results with a minumum accurancy of 90\%, we decide to repeat the same scenario with
the same parameters for 200 times, and than compute mean and median values from performance indexes, alogn with their confidence intervals.
\section{Big}\label{big}
\subsection{Coverage}
This plot show the coverage achived in function of P, for different values of R.
\begin{figure}[H]
    \begin{center}
        \scalebox{0.42}{\import{img}{Big_CovP_median.pgf}}
    \end{center}
    \vspace*{-1cm}
\end{figure}
As we can observe, the coverage we can reach is maximum when P is low; this was predictable by the fact that a low transmission
probability minimize the number of collisions, increasing the number of correct transmissions. 
The acheiveble coverage decrease with the increasing of p, but this beahveour is quite slow and
dont degenerate to 0, but it land at a value around (XX\%) of the bmaximum. %todo
This shape is common for all the values of the transmission range R, but it's more evident for R$>$10.
This plot show the coverage achived in function of R, for different values of P.
\begin{figure}[H]
    \begin{center}
        \scalebox{0.9}{\import{img}{Big_CovRange_mean.pgf}}
    \end{center}
    \vspace*{-1cm}
\end{figure}
As we can expect, and accordingly with the previus plot, the final coverage of the flooplan is very
influenced by the transmission range; in this plot we can observe that the covreage increase exponentially
in respect to the transmission range. As we can expect, when the transmission range become very large ($>20m$),
the coverage tend to 100\%; we compute a good fit for this exponential around: %todo

\subsection{Duration}
This plot show the Duration of the simulation (in slot) in function of P, for different values of R.
\begin{figure}[H]
    \begin{center}
        \scalebox{0.42}{\import{img}{Big_DurP_median.pgf}}
    \end{center}
    \vspace*{-1cm}
\end{figure}
This plot show the Duration of the simulation (in slot) in function of R, for different values of P.
\begin{figure}[H]
    \begin{center}
        \scalebox{0.9}{\import{img}{Big_DurRange_mean.pgf}}
    \end{center}
    \vspace*{-1cm}
\end{figure}

\subsection{Collisions}
This plot show the Number of collisions in function of P, for different values of R.
\begin{figure}[H]
    \begin{center}
        \scalebox{0.4}{\import{img}{Big_CollP_median.pgf}}
    \end{center}
    \vspace*{-1cm}
\end{figure}
This plot show the Number of collisions in function of R, for different values of P.
\begin{figure}[H]
    \begin{center}
        \scalebox{0.69}{\import{img}{Big_CollRange_mean.pgf}}
    \end{center}
    \vspace*{-1cm}
\end{figure}
