%-------------------------------------------------------------------------------
% File: appendicies.tex
%       Epidemic Broadcast project documentation.
%
% Author: Marco Pinna, Rambod Rahmani, Yuri Mazzuoli
%         Created on 05/12/2020
%-------------------------------------------------------------------------------
\chapter{Appendices}

\section*{Appendix A}
Given a scenario with $N$ transmitter devices and a target device $T$ in reach of all the transmitters, let us define the probabilities $P_{1}(j, N)$ as the probability of $j$ devices out of $N$ transmitting at the same time during slot 1 and $P_{i}(j)$ as the probability of $j$ devices transmitting at the same time during slot $i$.\\
By specification, the successful reception of the message by device T happens if and only if \textbf{one} of the transmitters sends the message during the slot.
Furthermore, the successful transmission of a device is ``\textit{a Bernoullian RV with success probability \emph{p} on every slot, until it achieves success}"; therefore we can model $P_{1}(j, N)$ as 
follows:

\[
\begin{drcases}
    &P_{1}(0, N) = (1-p)^{N} \\
    &P_{1}(1, N) = N p(1-p)^{N-1}\\
    &P_{1}(2, N) = {N\choose2}p^{2}(1-p)^{N-2}\\
	&P_{1}(3, N) = {N\choose3}p^{3}(1-p)^{N-3}\\
	&...\\
	&P_{1}(N-1, N) = {N\choose N-1}p^{N-1}(1-p)\\
	&P_{1}(N, N) = {N\choose N}p^{N}\\
\end{drcases}
P_{1}(j, N) = {N\choose j} p^{j} (1-p)^{N-j}
\]
\\
\\
As for $P_{i}(j)$ we can model the system as if it was in the first slot, with the total number of active devices now being equal to $N-t$, where $t$ is the number of devices that have transmitted in the $(i\text{-}1)$-th slot.

%Therefore, for $i=2$ we have:
%
%\begin{align*}
%    &P_{2}(0)
%    	\begin{aligned}[t]
%    		&= P_{1}(0, N)P_{1}(0, N) + P_{1}(2, N)P_{1}(0, N - 2) + ... + P_{1}(N-1, N)P_{1}(0, 1) = \\
%			&= P_{1}(0, N)P_{1}(0, N) + \sum_{k=2}^{N-1}P_{1}(k, N)P_{1}(0, N-k) = \\
%			&= \sum_{k=0}^{N-1}P_{1}(k, N-k)P_{1}(0, N) - P_{1}(1, N)P_{1}(0, N-1)
%		\end{aligned}\\
%	&P_{2}(1)
%    	\begin{aligned}[t]
%    		&= P_{1}(0, N)P_{1}(1, N) + P_{1}(2, N)P_{1}(1, N - 2) + ... + P_{1}(N-1, N)P_{1}(1, 1) = \\
%			&= P_{1}(0, N)P_{1}(1, N) + \sum_{k=2}^{N-1}P_{1}(k, N)P_{1}(1, N-k) = \\
%			&= \sum_{k=0}^{N-1}P_{1}(k, N)P_{1}(1, N - k) - P_{1}(1, N)P_{1}(1, N-1)
%		\end{aligned}\\	
%    &P_{2}(2)
%    	\begin{aligned}[t]
%    		&= P_{1}(0, N)P_{1}(2, N) + P_{1}(2, N)P_{1}(2, N - 2) + ... + P_{1}(N-2, N)P_{1}(2, 2) = \\
%			&= P_{1}(0, N)P_{1}(2, N) + \sum_{k=2}^{N-2}P_{1}(k, N)P_{1}(2, N-k) = \\
%			&= \sum_{k=0}^{N-2}P_{1}(k, N)P_{1}(2, N - k) - P_{1}(1, N)P_{1}(2, N-1)
%		\end{aligned}\\	
%	&P_{2}(3) = ... = \sum_{k=0}^{N-3}P_{1}(k, N)P_{1}(3, N - k) - P_{1}(1, N)P_{1}(3, N-1)\\
%	&...\\
%	&P_{2}(N-2) = \sum_{k=0}^{2}P_{1}(k, N)P_{1}(N-2, N - k) - P_{1}(1, N)P_{1}(N-2, N-1)\\
%	&P_{2}(N-1) = \sum_{k=0}^{1}P_{1}(k, N)P_{1}(N-1, N - k) - P_{1}(1, N)P_{1}(N-1, N-1)\\
%	&P_{2}(N) = P_{1}(0, N)P_{1}(N, N)
%\end{align*}
%
%
%which has the general form:
%  \begin{equation}\label{eq:1}
%    P_{2}(j) =
%    \begin{cases*}
%      \sum_{k=0}^{N-1}P_{1}(k, N)P_{1}(0, N - k) - P_{1}(1, N)P_{1}(0, N-1) & j = 0 \\
%      \sum_{k=0}^{N-j}P_{1}(k, N)P_{1}(j, N - k) - P_{1}(1, N)P_{1}(j, N-1) & 0 $<$ j $<$ N \\
%      P_{1}(0, N)P_{1}(N, N) & j = N \\
%    \end{cases*}
%  \end{equation}
%\\
%\\
%where the term with the minus sign is due to the fact that, if only one device transmitted during slot $i$, the target device T will have correctly received the message and therefore, starting from slot $i+1$ onwards, it will not be listening for incoming messages any more but it will be itself transmitting instead.\\
%\\

\section*{Appendix B}

An example of stochastic matrix for a system with $N=5$ and $p = 0.4$:


\begin{equation*}
P = 
\begin{bmatrix}
P_{0,0}	& P_{0,2}	& P_{0,3}  	& P_{0, 4}	& P_{0,S}	& P_{0,5} \\
		& P_{2,2}	& 0  		& P_{2, 4}	& P_{2,S}	& P_{2,5} \\
		& 			& P_{3,3}	& 0			& P_{3,S}	& P_{3,5} \\
		& 			& 			& P_{4,4}	& P_{4,S}	& 0\\
		& 			& 			& 			& 1			& 0\\
0		& 			& 		  	& 			& 			& 1\\
\end{bmatrix}
\label{exampleMatrix}
\end{equation*}

\hfill \break

Here with numerical values (rounded to 4 decimal places):

\begin{equation*}
P = 
\begin{bmatrix}
0.0778	& 0.3456	& 0.2304  	& 0.0768	& 0.2592	& 0.0102 \\
		& 0.216		& 0  		&0.288		& 0.432		& 0.064 \\
		& 			& 0.36		& 0			& 0.48		& 0.16 \\
		& 			& 			& 0.6		& 0.4		& 0\\
		& 			& 			& 			& 1			& 0\\
0		& 			& 		  	& 			& 			& 1\\
\end{bmatrix}
\label{exampleMatrixValues}
\end{equation*}

\hfill \break

\section*{Appendix C}
Given a generic convess shape as a floorplan, we can compute it'area ($A$); 
if we imagine to put a (puntiform) transmitter (trx) in a random point of the floorplan, 
it's easy to compute the probability for the trx to be located in a given 
point \textit{x} as:
\begin{equation*}
  P(x)=1/A
\end{equation*}
Given that the trasmssion raduis in $r$, we can define a circle centered in x, 
as the transmission range of the first trx ($R$). If the trx in far enough from 
the border of the floorplan, all it's transmission range fall into the floorplan; 
in this case, the probability for another (random positioned) trx to be placed 
inside this transmission range is equal to:
\begin{equation*}
  P(x') = \int_{R}^{} P(x)= \frac{2\pi r^2}{A} \ 
\end{equation*}
If the firts trx is placed at a distance less than $r$ from one or more borders of the floorplan,
the probaility to randomly place another trx in it's range is less than the previous case; this is 
because part of the area of the transmission range will fall out of the floorplan, and,
by construction, trxs can't be placed there. 
Depending on the floorplan' shape, we can identify the partition of the shape where the transmission range
will fall out of the shape itself, and we will do specific computatin for this area;
In general, by the middle value theorem for integrals, and the total probability theorem, the probailiy
for 2 transmitters (let $x$ and $y$ be their coordinates) randomly placed to be close enough to communicate it's:
\begin{equation*}
  \rho =P(|x-y|<r) = \frac{1}{A} \int_{A}^{} P(x') \ 
\end{equation*}
Since we know that $P(x')$ is not constant for all the area of the shape, this integral might be difficult to compute,
even for a square floorplan; in any case it's value, it's constant for a given floorplan and a given $r$.
Taking in to consideration only one trx, we can compute the probability density function for the number of trx in its
transmission range, in function of the number of trx in the entire floorplan $C(N)$. Every time we put
a trx in the floorplan, the probability that it falls in the transmission range of the highlighted one is $\rho$. This is true
for every $N > 1$, because of indipendance. 
\begin{equation*}
	C(N)= \left(\frac{1}{p}\right)^{\left( N-1 \right)}
  \end{equation*}
This is a Bernullian distribution, and we can use this result only taking one trx at time; in other words we can't use this 
distrubution for all the trx in a random configured floorplan, because they are not indipendent. 
The distrubution for one trx will be influenced by the position of others. 
Anyway, we can apply this distribution to many transmitters, each one taken from a different floorplan; in this way we can 
validate the indipendanceof different random configurations.\\
